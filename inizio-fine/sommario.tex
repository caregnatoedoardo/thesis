% !TEX encoding = UTF-8
% !TEX TS-program = pdflatex
% !TEX root = ../tesi.tex

%**************************************************************
% Sommario
%**************************************************************
\cleardoublepage
\phantomsection
\pdfbookmark{Sommario}{Sommario}
\begingroup
\let\clearpage\relax
\let\cleardoublepage\relax
\let\cleardoublepage\relax

\chapter*{Sommario}

Il presente documento descrive il lavoro svolto durante il periodo di stage, della durata di circa trecento ore, dal laureando \textbf{Edoardo Caregnato} presso l'azienda \textbf{PAT - Infinte Solutions}.
Gli obiettivi da raggiungere, concordati tra Azienda ed Università, al fine di completare con successo l'esperienza di stage erano questi di seguito esplicati.\\
In primo luogo è stato richiesto lo studio individuale relativo alle differenze architetturali tra \gls{Container} e \gls{Virtual Machine}, con relativa discussione ed esposizione di quanto elaborato al Tutor aziendale \textbf{Ruggero Maffei}.\\
In secondo luogo è stato richiesto uno studio individuale di \gls{Docker} e \gls{Docker Compose} e delle relative \gls{API} di automation, il quale scopo finale era quello di predisporre un ambiente totalmente compatibile al fine di eseguire con successo i due applicativi di punta dell'Azienda, ovvero \gls{HDA} e \gls{CX Studio}. Dopo un'attenta analisi sulla fattibilità e tempistiche del progetto, è stato concordato, assieme all'Azienda, di concentrare l'esperienza curricolare sulla containerizzazione dell'applicativo \gls{HDA} con tutte le sue relative estensioni.\\
Il terzo obiettivo dello stage curricolare è stata la predisposizione dei relativi container atti all'esecuzione dell'applicativo \gls{HDA} comprensivo di tutti gli strumenti di monitoraggio richiesti dall'Azienda.\\
Quarto ed ultimo obbiettivo è stato lo studio e la creazione di container atti all'aggiornamento della versione di \gls{HDA}, con relativo studio sulla possibilità di automazione dell'intero processo.

%\vfill
%
%\selectlanguage{english}
%\pdfbookmark{Abstract}{Abstract}
%\chapter*{Abstract}
%
%\selectlanguage{italian}

\endgroup			

\vfill

