% !TEX encoding = UTF-8
% !TEX TS-program = pdflatex
% !TEX root = ../tesi.tex

%**************************************************************
\chapter{Conclusioni}
\label{cap:conclusioni}
\intro{Breve introduzione al capitolo}
Il presente capitolo finale vuole esporre una panoramica generale sul raggiungimento generale degli obiettivi dello stage curricolare, valutando le conoscenze acquisite dallo stagista e la qualità generale dell'esperienza di stage effettuata all'interno dell'Azienda.
\section{Raggiungimento degli obiettivi}
Come riportato nel capitolo 2.2, causa mole di lavoro troppo impegnativa in rapporto al tempo già rilevato ad inizio del percorso di stage, sotto completo suggerimento del tutor aziendale Sig. Maffei Ruggero, è stato consigliato al candidato di concentrarsi esclusivamente sulla creazione di una immagine \textit{Dockerizzata} del solo prodotto HDA con le relative dipendenze come descritto nell'apposito capitolo prima citato.
Avendo l'Azienda stessa modificato già ad inizio stage la lista degli obiettivi obbligatori e desiderabili che il candidato avrebbe dovuto portare a termine, lo stagista è riuscito a portare a termine ed a soddisfare \textbf{tutti} gli obiettivi ed i requisiti con una settimana d'anticipo, \textbf{ad eccezione} del requisito che richiedeva la creazione di un container per l'esecuzione degli \textbf{unit test} automatici delle immagini create. La tabella sottostante rappresenta, graficamente, gli obiettivi e requisiti con il relativo stato di raggiungimento (R) o non raggiungimento (NR):
\newpage
\begin{center}
\begin{tabular}{|m{25em} | m{2,1cm} | m {1cm}|} 
 \hline
 \textbf{Descrizione requisito} & \textbf{Quantitativo orario (h)} & \textbf{Stato}\\ [0.5ex] 
 \hline\hline
 Introduzione al tema Container vs Virtual machine: differenze tra le due tecnologie e
illustrazione dei vantaggi derivanti dall'adozione dei Container & 40 & R\\ 
 \hline
 Docker ed estensioni (Docker, Compose e Swarm, Kubernetes) e relative API di
automation: analisi delle componenti dell'ecosistema e delle opportunità di utilizzarle
ai fini progettuali & 40 & R\\
 \hline
 Approfondimento sull'architettura su due casi studi da containerizzare: declinazione
della soluzione tecnologica identificata ai punti precedenti su due applicazioni PAT & 120 & R\\
 \hline
 Creazione dei container ed automatizzazione del processo di building: realizzazione
del processo di creazione delle immagini ed automazione dello stesso & 80 & R\\
 \hline
 Utilizzo di un container per verifica dell'esecuzione degli unit test: avvio delle
immagini per effettuazione degli unit test in automatico & 40 & NR\\ [1ex] 
 \hline
 \textbf{TOTALE} & \textbf{320} & \textbf{4/5}\\
 \hline
\end{tabular}
\end{center}
La settimana avanzata è stata usata dal candidato per la scrittura di tutta la documentazione di supporto interna all'Azienda, per permettere ad eventuali dipendenti di capire ed ampliare in futuro quanto sviluppato durante tutta l'esperienza di stage. La documentazione redatta è stata inserita nell'apposita sezione OneNote dell'Azienda assieme a tutta la documentazione interna dei loro prodotti.

%**************************************************************
\section{Conoscenze acquisite}
Le conoscenze conseguite a seguito dell'esperienza di stage sono state molteplici, in primis tra tutte la tecnologia che ho implementato. L'aver studiato in maniera approfondita la tecnologia Docker mi ha permesso di rendermi conto dell'esistenza di una tecnologia completamente alternativa, e per molti aspetti altrettanto valida, alle classiche architetture a macchine virtuali, con conseguente mia possibilità di ulteriore approfondimento e studio individuale futuro nell'ottica di poter proporre, alle future aziende con le quali collaborerò, delle soluzioni architetturali non più basate solo ed esclusivamente sulle virtual machine.
A livello umano, ho trovato persone estremamente preparate e professionali con cui sono tutt'ora in contatto. Queste persone hanno contribuito alla mia crescita interiore facendomi rendere conto che, in un clima di reciproca fiducia e collaborazione, qualsiasi progetto, se ben supportato da colleghi attenti e preparati, potrà essere sviluppato con successo senza un rischio elevato di fallimento immediato ed a lungo termine. Avendo lavorato in un team così dinamico, ho inoltre rivisto la mia metodologia di lavoro, tipicamente molto autonoma, nell'ottica di una mia sempre più aperta visione ad eventuali osservazioni ed insegnamenti dettati da dipendenti con molta più preparazione ed esperienza alle loro spalle rispetto al sottoscritto. Questi insegnamenti rientreranno senz'altro nel mio bagaglio culturale, nella speranza di \textit{saperli} e \textit{poterli} attuare in ulteriori esperienze lavorative.

%**************************************************************
\section{Valutazione personale}
La mia personale opinione su questa esperienza di stage non può che essere assolutamente positiva. Ho avuto l'opportunità di lavorare a contatto con un team altamente qualificato, e di conoscere persone molto competenti nel loro ambito che spero continuino a far parte della mia rete di amicizie. L'aver fatto questa esperienza di stage mi ha, inoltre, permesso di apprendere com'è organizzata internamente un'azienda informatica di medie dimensioni. L'avermi dovuto rapportare con diverse persone di grado differente ha fatto sì che abbia potuto mettere alla prova la mia capacità di relazionarmi con le persone, diversificando, naturalmente, la mia metodologia di interlocuzione a seconda del ruolo del componente dell'Azienda con cui mi stavo confrontando.\\
Sempre grazie a questa esperienza di tirocinio, ho imparato a gestire il mio tempo in maniera nettamente migliore rispetto a tutto il percorso universitario fatto. Il dovermi rapportare con scadenze settimanali e vincoli, obbligatori e non, hanno contribuito a sviluppare in me una capacità migliore di gestione temporale, facendomi imparare, in primis, che il tempo utilizzato per lo studio di una particolare tecnologia è forse la parte più importante di un progetto stesso, e che quindi va eseguita senza alcuna fretta o pregiudizio.







