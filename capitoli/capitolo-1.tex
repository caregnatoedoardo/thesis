% !TEX encoding = UTF-8
% !TEX TS-program = pdflatex
% !TEX root = ../tesi.tex

%**************************************************************
\chapter{Introduzione}
\label{cap:introduzione}
%**************************************************************

Il seguente capitolo intende introdurre il lettore al \textbf{contesto applicativo} dello stage effettuato presso \textbf{PAT Group} evidenziando, in ordine, le \textbf{metodologie di lavoro} dell'Azienda, l'idea generale alla base dello stage formativo e l'\textbf{organizzazione} del presente documento.

%\noindent Esempio di utilizzo di un termine nel glossario \\
%\gls{api}. \\

%\noindent Esempio di citazione in linea \\
%\cite{site:agile-manifesto}. \\

%\noindent Esempio di citazione nel pie' di pagina \\
%citazione\footcite{womak:lean-thinking} \\

%**************************************************************
\section{L'azienda}

\textbf{Pat Group} è un'azienda innovativa che da oltre 25 anni opera nello sviluppo di applicativi software per imprese ed istituzioni pubbliche e private. Il suo principale \textit{core business} comprende lo sviluppo di applicativi di \textit{IT Service Management}, \textit{CRM}, \textit{lead management}, \textit{automazione dei processi di business} e \textit{social collaboration}.
Le principali energie che l'Azienda mette a disposizione sono destinate all'\textbf{evoluzione dei processi di business} dei clienti, fornendo supporto per l'innovazione digitale e l'automazione dei processi interni, per far raggiungere ad ogni cliente la \textit{leadership} nel relativo settore in cui opera.\\
I principali valori che PAT desidera ottenere sono riassumibili nei seguenti punti:
\begin{itemize}
	\item \textbf{Successo:} l'Azienda lavora costantemente al fine del raggiungimento, da parte dei suoi clienti, degli obiettivi di successo;
	\item \textbf{Innovazione:} PAT accompagna i propri clienti attraverso la difficile fase del cambiamento digitale, spingendo a innovare e costantemente aggiornare le tecnologie interne;
	\item \textbf{Qualità:} l'impegno di PAT, verso qualsiasi dei suoi clienti, è quello del raggiungimento di questo valore attraverso procedure di innovazione riconosciute e certificate a loro volta da standard qualitativi;
	\item \textbf{Talento:} PAT organizza per i suoi clienti dei percorsi formativi interni mirati alla ricerca dei talenti di ogni collaboratore;
	\item \textbf{Fiducia:} la realtà aziendale è fortemente concentrata sulla fiducia nella professionalità dei collaboratori dell'Azienda, per guidare al meglio nella crescita professionale ogni cliente;
	\newpage
	\item \textbf{Trasparenza:} al fine di raggiungere ogni obiettivo e valore l'Azienda adotta un metodo di comunicazione costante e continuo, atto a monitorare tutti i progressi effettuati ed interventire tempestivamente in casi di criticità.
\end{itemize}
PAT opera all'interno di un luogo innovativo chiamato \textbf{InfiniteArea}, ovvero uno spazio dedicato alla crescita di idee innovative nell'ottica di una futura implementazione nel \textit{core business} dell'Azienda. \\
Da giugno 2013, PAT entra a far parte dell'universo \textbf{Zucchetti}, primo gruppo italiano nel panorama ICT, con \textbf{Patrizio Bof} in qualità di fondatore ed unico amministratore della Società \textbf{PAT Group}.\\
L'Azienda collabora e fornisce soluzioni software per una moltitudine di clienti, alcuni tra i più importanti risultano \textbf{Pirelli}, \textbf{Aruba}, \textbf{BPER}, \textbf{AirDolomiti}, \textbf{ARVAL} e \textbf{WeBank}.
%**************************************************************
\section{Organizzazione dei capitoli}
Il seguente elenco intende fornire al lettore, in maniera ordinata, una panoramica informativa riassuntiva circa il contenuto e la suddivisione in capitoli di questo documento:
\begin{description}
    \item[{\hyperref[cap:descrizione-stage]{capitolo 2:}}] in questo capitolo viene esposta una \textbf{panoramica generale} relativa al progetto assegnato;
    
    \item[{\hyperref[cap:container-vm]{capitolo 3:}}] questo capitolo tratta le principali differenze tra \textbf{architettura a macchine virtuali ed a container}, fornendo al lettore un approfondimento dettagliato su ognuna delle tecnologie trattate;
    
    \item[{\hyperref[cap:architettura-progetto-container]{capitolo 4:}}] questo capitolo espone l'\textbf{architettura generale} del progetto implementata durante tutto il percorso di stage;
    
    \item[{\hyperref[cap:orchestrazione-container]{capitolo 5:}}] questo capitolo approfondisce la metodologia di \textbf{orchestrazione tra container} tramite utilizzo di Docker Compose;
    
    \item[{\hyperref[cap:nginx-reverse-proxy]{capitolo 6:}}] in questo capitolo viene trattato il tema di \textbf{Reverse Proxy}, fornendo al lettore tutti i passi eseguiti al fine dell'implementazione sul progetto assegnato;
    
    \item[{\hyperref[cap:conclusioni]{capitolo 7:}}] questo capitolo conclusivo espone una \textbf{panoramica sugli obiettivi raggiunti}, il grado di conoscenze acquisite e grado di soddisfacimento relativo allo stage curricolare da parte dello studente. 
\end{description}