% !TEX encoding = UTF-8
% !TEX TS-program = pdflatex
% !TEX root = ../tesi.tex

%**************************************************************
\chapter{Container VS Virtual Machine}
\label{cap:container-vm}
%**************************************************************

\intro{Introduzione al capitolo}\\
Nel presente capitolo si esporranno le principali differenze tra un'architettura basata su macchine virtuali ed un'altra basata invece su container.

%**************************************************************
\section{Differenze architetturali tra Container e VM}
La \textbf{virtualizzazione} è un insieme di software in grado di astrarre componenti \gls{hardware}, permettendo l'esecuzione, anche simultanea, di più \gls{sistemi operativi} su un singolo \gls{client}. Verso la fine degli anni '90, la virtualizzazione ha cominciato ad essere sempre più utilizzata in ambienti \textit{enterprise}, permettendo un aumento di scalabilità e flessibilità dell'infrastruttura informatica aziendale riducendone notevolmente i costi di gestione\footcite{fonte: https://www.vmware.com/it/solutions/virtualization.html}. I \textbf{vantaggi} legati all'utilizzo della virtualizzazione, nello specifico, tramite l'uso di una o più macchine virtuali, comportano una separazione tra il sistema operativo \gls{host} e \gls{guest}, fornendo una serie di accessi logici utilizzati da utenti esterni agli applicativi eseguiti in ogni macchina virtuale. \\Oltre ad una esecuzione parallela, dal punto di vista dell' \textit{host}, di molteplici applicativi, un'architettura a VM è più facilmente \textbf{manutenibile}: una macchina virtuale infatti, può essere facilmente aggiornata, avviata o arrestata in base alle esigenze di carico (ex: \textit{load-balancing}) o aziendali.
La virtualizzazione, inoltre, aumenta l'\textbf{affidabilità} dell'intero sistema, in quanto garantisce l'\textbf{isolamento} di programmi e servizi i quali non andranno in conflitto tra di loro, contenendo, in aggiunta, il numero di server fisici presenti in \gls{datacenter} nel caso in cui molteplici macchine virtuali vengano eseguite su un singolo \textit{host}, con conseguente notevole riduzione dei costi.
Un ulteriore vantaggio della virtualizzazione si rivela nel caso di \textit{\gls{disaster recovery}}, dove l'intero sistema operativo \textit{guest} può essere facilmente ripristinato su un altro server, indipendentemente dall'\textit{hardware}, riducendo così notevolmente i tempi di indisponibilità di servizio (\textit{downtime}) in caso di guasto favorendo una maggior facile e rapida procedura di data \textit{\gls{recovery}}.
Esistono diversi tipi di virtualizzazione: \textbf{\textit{native}} e \textbf{\textit{hosted}}.
Una virtualizzazione di tipo \textbf{native} si appoggia direttamente all'\textit{hardware} \textit{host}, controllandolo direttamente per garantire tutte le funzionalità della virtualizzazione, come ad esempio \textbf{\gls{Hyper-V}} della Microsoft\footcite{questa funzionalita' e' presente solamente nelle versioni Pro e Server di Windows 10} oppure l'applicativo \textbf{\gls{Xen}} ampiamente utilizzato anche nell'ambiente Cloud di Amazon.
La virtualizzazione di tipo \textbf{hosted} è invece in esecuzione sul sistema operativo \textit{host} senza alcuna interfaccia diretta con l'hardware del computer. Questo tipo di virtualizzazione è molto diffusa, in quanto permette di accedere, in una maniera semplice ed immediata, al sistema operativo \textit{host} e \textit{guest} in maniera simultanea. Gli applicativi più usati in ambito enterprise che usano un tipo di virtualizzazione \textit{hosted} sono, ad esempio, \textit{VMware} o \textit{VirtualBox}.\\
Nella seguente figura è rappresentato un sistema operativo Windows 10 pro virtualizzato tramite virtualizzazione \textit{native} tramite software Hyper-V:
%TODO: Esempio di Windows 10 Pro virtualizzato tramite Hyper-V
\\Di seguito, un esempio di sistema operativo Windows 10 Pro virtualizzato tramite virtualizzazione \textit{hosted} tramite le due soluzioni software appena descritte:
%TODO: Esempio di schermata di VMware e VirtualBox con, in esecuzione, Windows 10
\\Al fine di permettere al sistema operativo \textit{host} la virtualizzazione di uno o più sistemi operativi, è necessario installare un \textit{\gls{hypervisor}}\footcite{installabile solamente se il processore supporta la virtualizzazione e se quest'ultima e' abilitata da BIOS}, \textit{native} o \textit{hosted}, ovvero uno strato software che si interfacci e gestisca tutte le istanze di macchine virtuali in esecuzione sulla macchina locale. 
\\La virtualizzazione non è priva di svantaggi. Il primo tra tutti, è appunto la necessità di dover \textit{virtualizzare} un intero sistema operativo al fine di eseguire l'applicativo virtuale desiderato.
Questo vincolo obbligatorio implica un consumo di memoria \gls{RAM} e di \gls{storage} non indifferente anche solo per eseguire il singolo sistema operativo virtualizzato senza alcuna applicazione virtuale in esecuzione.
Ne consegue quindi, che un'architettura a macchine virtuali avrà bisogno di uno spazio di \textit{storage} e di un quantitativo di memoria \textit{RAM}\footcite{il tipo di RAM "ECC" risulta preferibile ma non obbligatorio} installata sul server non indifferente. Anche in termini di consumo \textit{\gls{CPU}}, la virtualizzazione di molteplici sistemi operativi con le relative applicazioni virtualizzate in esecuzione può comportare grossi carichi prestazionali al server fisico, in quanto la CPU dell'host dovrà servire ed eseguire ogni sistema operativo di ogni istanza di virtualizzazione.\\
Dal punto di vista della sicurezza, quando si virtualizza un sistema operativo, sia nella virtualizzazione \textit{native} che \textit{hosted}, alcuni registri CPU sono direttamente esposti alla macchina virtuale come, ad esempio, i registri \textbf{VT-x} e \textbf{VT-d} del processore\footcite{e' necessario abilitare le estensioni di virtualizzazione da BIOS della scheda madre.} \footcite{nel caso di architettura avente processori Intel, IOMMU per architetture basate su processori AMD.}. Questi registri permettono al processore di non rendere accessibile la totalità dei suoi registri all'hypervisor e di controllare le chiamate dirette al \textit{\gls{DMA}} da parte delle soluzioni software virtualizzate. \\
Relativamente alla condivisione della rete tra macchine virtuali e host fisico, nel caso in cui si fosse installato un commutatore di rete virtuale di tipo \textbf{\gls{NAT}}, la scheda di rete dell'host e il relativo traffico sarebbe esposta a tutto il set applicativo virtualizzato e viceversa, con conseguente mancante isolamento tra macchine virtuali stesse ed host fisico. Ne conseguirebbe quindi, che eventuali condivisioni di rete, o connessioni applicative, sarebbero disponibili a tutto il set di macchine virtuali. Una possibile soluzione a questo problema potrebbe essere il passaggio da commutatore virtuale \textit{NAT} ad un tipo di commutatore virtuale che riesca ad isolare le singole macchine virtuali tra di esse e l'host fisico, anche, nel caso più estremo, assegnando ad ogni macchina virtuale una propria scheda di rete ed una propria \textit{\gls{VLAN}} di rete dedicata\footcite{per creare o impostare una VLAN, fare riferimento al router/firewall o allo switch di rete}.\\
Virtualizzare un intero sistema operativo implica, come abbiamo appena analizzato, un elevato consumo di risorse fisiche, specialmente nel caso in cui, per esigenze lavorative, si debba ricorrere ad una multipla virtualizzazione di sistemi operativi dove, in ognuno, viene eseguita una specifica applicazione che deve essere accessibile ad altri \textit{client}.
Uno dei principali aspetti positivi di un'architettura a container sta proprio nel poter virtualizzare (o \textit{containerizzare} nel caso appunto di container) una singola e specifica applicazione senza la necessità di inglobare un intero sistema operativo nell'immagine virtuale. Ne consegue quindi, che il container applicativo risultante di un'applicazione \textit{containerizzata} è di gran lunga di dimensione inferiore rispetto all' immagine\footcite{inteso come dimensione in Gb del virtual disk image (*.vdi) dell'immagine virtualizzata} della stessa applicazione \textit{virtualizzata}.\\
Un container è quindi una singola unità atomica contenente l'applicativo (il programma \textit{containerizzato}) con i relativi file atti alla sua corretta esecuzione senza l'immagine di un sistema operativo completo.
Al momento dell'esecuzione del container, l'applicazione \textit{containerizzata} verrà eseguita immediatamente sopra lo stato del sistema operativo host, attraverso l'aiuto del Docker Engine, senza alcun hypervisor come, ad esempio, nel caso dell'architettura a macchine virtuali.
Un'architettura a \textbf{container} infatti, a differenza dell'architettura a macchine virtuali, garantisce un'esecuzione separata e protetta di ogni singolo applicativo compatibile con il sistema operativo host, indipendentemente dal numero di container presenti nel sistema o dal tipo di interfaccia di rete. E' possibile, inoltre, far coesistere multipli container di uno stesso applicativo in esecuzione nello stesso momento (anche sfruttando il \textit{load-balancing}, come si accennerà nel corso di questa tesi) assegnandoci, esattamente come con le macchine virtuali, eventuali indirizzi IP statici, CPU limit e disk quota.
Essendo un container una \textbf{\gls{sandbox}} applicativa indipendente dal sistema operativo, i dati generati dalla sua esecuzione sono destinati a scomparire nell'eventualità in cui il container venisse distrutto. Per ovviare a questo problema, si può ricorrere ad una tecnica di volume-mapping, ovvero una tecnica che permette di esporre il \textbf{\gls{filesystem}} interno al container permettendone quindi la lettura e scrittura direttamente da parte dell'host. La tecnica appena accennata sarà trattata in maniera più approfondita nel corso della lettura di questa tesi.
Un altro dei vantaggi decisivi di un'architettura a container è la sua facilità di gestione. L'avvio, rimozione o la duplica dei container è un'operazione relativamente meno onerosa rispetto alla controparte nelle macchine virtuali (basti solo pensare al tempo di \textit{\gls{boot}} del sistema operativo), e può essere facilmente automatizzata e gestita dal Docker Engine. Ne consegue quindi che la scalabilità, ovvero la facilità di modifica dell'infrastruttura per far fronte alle variazioni di mole di informazioni trattate o carichi di lavoro, risulta di gestione più semplice anche per la figura sistemistica interna all'azienda.

%TODO: come si ottiene un container (immagine s.o. + applicazione) vs come si ottiene lo stesso con una macchina virtuale.


\section{Creazione di container vs creazione di VM}







