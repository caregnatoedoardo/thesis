% !TEX encoding = UTF-8
% !TEX TS-program = pdflatex
% !TEX root = ../tesi.tex

%**************************************************************
\chapter{Descrizione dello stage}
\label{cap:descrizione-stage}
%**************************************************************

\intro{Breve introduzione al capitolo}\\
In questo capitolo si esporrà una panoramica relativa al progetto assegnato dall'Azienda, con un approfondimento relativo ai requisiti obbligatori da soddisfare ed obiettivi raggiunti. 

%**************************************************************
\section{Scopo dello stage curricolare}
Lo scopo principale dello stage è la containerizzazione delle due soluzioni applicative maggiormente utilizzate dall'Azienda, ovvero \gls{HDA} e \gls{CX Studio} in ambito Windows, con la possibilità di monitoraggio in real-time delle stesse tramite containerizzazione di ulteriori applicativi quali "Telegraf", "Loki" e "Promtail" precedentemente configurati da un altro stagista.
Tramite containerizzazione dei due applicativi sopracitati è stato richiesto, in aggiunta, la creazione di uno script di reverse-proxy automatizzato che, tramite Docker API, identificava le istanze containerizzate di HDA attualmente in esecuzione ed aggiornava il file di configurazione di NGINX per permettere l'accesso a delle utenze esterne ai container in esecuzione sull'host.\\
Per raggiungere questo obiettivo, lo stagista ha dovuto apprendere in maniera approfondita le tecnologie di containerizzazione e di orchestrazione, quali \gls{Docker} e \gls{Docker Compose}, unite ad elementi di networking generale e scripting in Powershell e cmd. Verso la conclusione dello stage, lo stagista ha dovuto studiare la soluzione di reverse proxy "NGINX", editandone, dapprima manualmente, tutti i file di configurazione.


%**************************************************************
\section{Requisiti e obiettivi}
Come già spiegato nell'introduzione di questo documento, causa mancanza di tempo e sotto totale suggerimento dell'Azienda, lo stagista si è dedicato alla creazione dell'immagine containerizzata del solo prodotto "HDA", escludendo quindi il software "CX Studio", creando, in aggiunta, quattro diversi container aggiuntivi per permettere una soluzione di reverse-proxy e di monitoraggio in real time dello stack applicativo containerizzato.
Causa questa modifica, i requisiti ed obiettivi dello stage curricolare sono così cambiati:
\begin{itemize}
	\item Studio dell'architettura dello stack applicativo delle soluzioni software aziendali;
	\item Studio delle soluzioni software di containerizzazione, con focus maggiore sull'applicazione "Docker";
	\item Creazione del primo container contenente un'immagine di HDA funzionante ed usabile;
	\item Creazione di un container atto all'aggiornamento del database dell'applicativo HDA in caso di aggiornamento da una versione legacy di HDA ad una versione 11.x.x;
	\item Creazione, configurazione ed orchestrazione di un container dedicato all'applicativo "LokI";
	\item Creazione, configurazione ed orchestrazione di un container dedicato all'applicativo "Promtail"
	\item Aggiornamento del container contenente l'immagine di HDA con aggiunta e configurazione dell'applicativo "Telegraf";
	\item Studio delle Docker API per riuscire ad avere la lista dei container attivi in un dato host;
	\item Creazione di uno script automatizzato che, tramite Docker API, ottiene le sandbox applicative di container attivi per la costruzione automatizzata del file di configurazione di NGINX (nginx.conf).
\end{itemize}

Tutti i requisiti sopra-citati sono stati completamente raggiunti e collaudati, come verrà spiegato successivamente nel corso di questo documento.

%**************************************************************
\section{Pianificazione}
La pianificazione atta al soddisfacimento dei requisiti scritti è stata strettamente dettata dall'Azienda, ed è stata seguita in maniera pedissequa durante tutto il corso dell'esperienza di stage.
Tutte le attività pianificate, con il rispettivo corrispettivo orario, possono essere riassunte nella seguente tabella:
\begin{center}
\begin{tabular}{|m{30em} | m{3,5cm}|} 
 \hline
 Descrizione & Quantitativo orario (h)\\ [0.5ex] 
 \hline\hline
 Introduzione al tema Container vs Virtual machine: differenze tra le due tecnologie e
illustrazione dei vantaggi derivanti dall'adozione dei Container & 40 \\ 
 \hline
 Docker ed estensioni (Docker, Compose e Swarm, Kubernetes) e relative API di
automation: analisi delle componenti dell'ecosistema e delle opportunità di utilizzarle
ai fini progettuali & 40 \\
 \hline
 Approfondimento sull'architettura su due casi studi da containerizzare: declinazione
della soluzione tecnologica identificata ai punti precedenti su due applicazioni PAT & 120 \\
 \hline
 Creazione dei container ed automatizzazione del processo di building: realizzazione
del processo di creazione delle immagini ed automazione dello stesso & 80 \\
 \hline
 Utilizzo di un container per verifica dell'esecuzione degli unit test: avvio delle
immagini per effettuazione degli unit test in automatico & 40 \\ [1ex] 
 \hline
 \textbf{TOTALE ORARIO} & \textbf{320}\\
 \hline
\end{tabular}
\end{center}