% !TEX encoding = UTF-8
% !TEX TS-program = pdflatex
% !TEX root = ../tesi.tex

%**************************************************************
\chapter{Descrizione dello stage}
\label{cap:descrizione-stage}
%**************************************************************

\intro{Breve introduzione al capitolo}\\
In questo capitolo si esporrà una panoramica relativa al progetto assegnato dall'Azienda, con un approfondimento relativo ai requisiti obbligatori da soddisfare ed obiettivi raggiunti. 

%**************************************************************
\section{Scopo dello stage curricolare}
Lo scopo principale dello stage è la containerizzazione delle due soluzioni applicative maggiormente utilizzate dall'Azienda, ovvero \gls{HDA} e \gls{CX Studio} in ambito Windows, con la possibilità di monitoraggio in real-time delle stesse tramite containerizzazione di ulteriori applicativi quali "Telegraf", "Loki" e "Promtail" precedentemente configurati da un altro stagista.
Tramite containerizzazione dei due applicativi sopracitati è stato richiesto, in aggiunta, la creazione di uno script di reverse-proxy automatizzato che, tramite Docker API, identificava le istanze containerizzate di HDA attualmente in esecuzione ed aggiornava il file di configurazione di NGINX per permettere l'accesso a delle utenze esterne ai container in esecuzione sull'host.\\
Per raggiungere questo obiettivo, lo stagista ha dovuto apprendere in maniera approfondita le tecnologie di containerizzazione e di orchestrazione, quali \gls{Docker} e \gls{Docker Compose}, unite ad elementi di networking generale e scripting in Powershell e cmd. Verso la conclusione dello stage, lo stagista ha dovuto studiare la soluzione di reverse proxy "NGINX", editandone, dapprima manualmente, tutti i file di configurazione.


%**************************************************************
\section{Requisiti e obiettivi}
Come già spiegato nell'introduzione di questo documento, causa mancanza di tempo e sotto totale suggerimento dell'Azienda, lo stagista si è dedicato alla creazione dell'immagine containerizzata del solo prodotto "HDA", escludendo quindi il software "CX Studio", creando, in aggiunta, quattro diversi container aggiuntivi per permettere una soluzione di reverse-proxy e di monitoraggio in real time dello stack applicativo containerizzato.

%**************************************************************
\section{Pianificazione}