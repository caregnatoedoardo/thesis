% !TEX encoding = UTF-8
% !TEX TS-program = pdflatex
% !TEX root = ../tesi.tex

%**************************************************************
\chapter{Reverse Proxy tramite NGINX sulle sandbox di HDA}
\label{cap:nginx-reverse-proxy}
\intro{Breve introduzione al capitolo}\\
In questo capitolo verrà esplicata la metodologia di \textbf{reverse-proxy} tra \textit{sandbox} applicative di HDA per premettere a diverse utenze esterne un accesso privato ad una singola istanza di HDA in esecuzione su un determinato host.

\section{Introduzione ad NGINX}
NGINX è un applicativo web-server multipiattaforma ad alte prestazioni, comunemente usato come reverse-proxy o load-balancer. Nato nel 2004 e progettato per garantire un basso consumo di memoria, utilizza un approccio asincrono, basato su \textit{event}, dove le richieste vengono gestite su un singolo thread. Ogni \textit{thread} worker, ovvero un processo che esegue un'elaborazione effettiva, è gestito da un \textit{thread} master, il quale lo coordina e lo controlla.\\
All'interno di Docker-Hub è presente un repository ufficiale, gestito dalla Nginx Foundations, contenente un'immagine ufficiale di NGINX basata sul sistema operativo Alpine Linux, ovvero una distribuzione Linux che ha fatto della leggerezza un requisito fondamentale. Come precedentemente spiegato in questo documento, il container atto al reverse-proxy automatizzato contenente un'immagine di NGINX è chiamato "nginxcontainer", e fa parte anch'esso della \textit{sandbox} applicativa di HDA.

\section{Logica di reverse-proxy su sandbox di HDA}
Affinchè un utente, interno od esterno alla \textit{intranet} aziendale, possa accedere al relativo container contenente l'istanza, personalizzata o meno, di HDA, è necessario che conosca il relativo \textbf{FQDN} della \textit{sandbox} applicativa di HDA alla quale si vuole accedere.\\
Un FQDN, acronimo di \textit{Fully Qualified Domain Name}, è un nome di dominio che specifica la posizione assoluta di un nodo (in questo caso, la nostra \textit{sandbox} applicativa di HDA) all'interno della gerarchia DNS. Un FQDN si distingue da un generico nome di dominio per l'aggiunta del nome dell'host nel prefisso della stringa di dominio, in modo tale da renderla univoca. Un esempio di FQDN di un container è facilmente visibile nell'immagine esplicante l'architettura del progetto nel capitolo 4.1.
Nonostante l'utente \textbf{non acceda in maniera diretta} al container contenente l'istanza di HDA, l'attributo FQDN di ogni container nelle relative \textit{sandbox} è assegnato in maniera automatica da Docker Compose secondo la seguente logica decisa assieme all'Azienda:
\centerline{\textbf{nomecliente\_nomecontainer\_numeroistanza}}
dove:
\begin{itemize}
	\item \textbf{nomecliente} identifica il nome della \textit{sandbox} applicativa a cui il container appartiene. Generalmente, il nome della \textit{sandbox} è il nome del cliente stesso;
	\item \textbf{nomecontainer} identifica il nome effettivo del container (ex: "hdabasecontainer") interno alla \textit{sandbox} applicativa avente il nome del cliente;
	\item \textbf{numeroistanza} nell'immagine 4.1 identificato con il simbolo "\#", identifica la quantità di istanze simili di un determinato container sono in esecuzione contemporaneamente sulla \textit{sandbox} applicativa.
\end{itemize}
Una nomenclatura di container fissa e \textit{standardizzata} secondo quanto deciso con l'Azienda è di fondamentale importanza, in quanto permette di instradare le richieste provenienti dall'esterno ad una qualsiasi \textit{sandbox} applicativa di HDA.
Per capire al meglio come questo processo funziona, è doveroso fornire in primis al lettore una panoramica grafica circa il funzionamento, con relativa spiegazione immediatamente sottostante:
\begin{figure}[!h]     
\centering 
    \includegraphics[width=1.0\columnwidth]{immagini/img/sandbox_nginx} 
    \caption{Rappresentazione grafica circa la gestione runtime delle \textit{sandbox} applicative di HDA}
\end{figure}\\
\newpage
Quanto sopra rappresentato, può essere facilmente riassunto nei seguenti passi:
\begin{enumerate}
	\item Arrivo della richiesta HTTP esterna (ex: cliente1.pat.it) all'NGINX dell'\textbf{host \textit{PAT.IT}};
	\item Il sistema operativo dell'host \textit{PAT.IT} controlla se l'FQDN della richiesta è presente all'interno del suo \textbf{file host};
	\item Il webserver NGINX controlla se l'entry "cliente1.pat.it" è presente nel suo \textbf{config file} (nginx.conf) e, se presente, provvede ad \textbf{inoltrare la richiesta} alla relativa \textit{sandbox};
	\item La richiesta HTTP \textbf{arriva al webserver NGINX interno} alla relativa \textit{sandbox} avente il nome uguale al prefisso della richiesta HTTP (ex: cliente1);
    	\item Il webserver NGINX interno alla \textit{sandbox} applicativa \textbf{inoltra la richiesta} HTTP al container "\textbf{hdabasecontainer}", permettendo quindi agli utenti dell'azienda "cliente1" il \textbf{totale accesso all'istanza di HDA};
	\item Eventuali programmi esterni di monitoring in real-time (quali, ad esempio, Grafana) possono ora essere collegati alla \textit{sandbox} applicativa semplicemente digitando nel relativo config-file (ex in Grafana: grafana.conf) l'\textbf{FQDN della relativa \textit{sandbox} da monitorare} (ex: cliente1.pat.it). Sarà compito del relativo NGINX interno alla \textit{sandbox} gestire e dirottare (tramite proprio file di configurazione nginx.conf) ai relativi container le richieste del programma di monitoring.
\end{enumerate}

\subsection{Aggiornamento di HDA nella sandbox applicativa}
Come precedentemente accennato in questo documento, e come rappresentato sempre nell'immagine 6.1, risulta facilmente intuibile la fattibilità di un eventuale aggiornamento di una relativa \textit{sandbox} applicativa. Per effettuare infatti un aggiornamento alla \textit{sandbox} applicativa di HDA, sarà sufficiente eseguire i seguenti passi:
\begin{enumerate}
	\item \textbf{Eliminazione} della \textit{sandbox} applicativa contenente la versione legacy di HDA;
	\item \textbf{Creazione} delle nuove immagini dei nuovi container della futura nuova \textit{sandbox} applicativa di HDA tramite batch-file \textbf{HDA\_sandbox\_builder.bat};
	\item Effettuare la \textbf{migrazione manuale} di eventuali \textit{overrides} o \textit{extensions} contenuti nella cartella App\_Data nel relativo volume-mapping nell'host ("hdashared");
	\item \textbf{Avviare}, tramite comando Docker Compose, la nuova \textit{sandbox} di HDA con il \textbf{nome uguale} alla \textit{sandbox} appena sostituita.
\end{enumerate}

Facendo questo, in maniera del tutto automatica, gli utenti accederanno così alla nuova istanza di HDA aggiornata.\\

Grazie al risolutore DNS interno a Docker ed al suo load-balancing automatico, nel caso di necessità di aggiornamento di \textbf{molteplici container} all'interno della \textit{sandbox} applicativa di HDA, si potrà aggiornare relativamente la versione di HDA in ogni container \textbf{senza generare alcun downtime} al cliente, semplicemente aggiornando \textbf{un container alla volta}, per permettere quindi al load-balancer di Docker di poter inoltrare tutte le richieste HTTP ad almeno un container attivo e funzionante di HDA. Il meccanismo di aggiornamento di multipli container all'interno di una singola \textit{sandbox} applicativa non era nei requisiti di questo stage né è stato in alcun modo testato dallo stagista durante tutto il periodo di stage. Questa possibilità di aggiornamento e di load-balancing dei container di HDA è stata sviluppata a livello completamente teorico dal sottoscritto e dal proprio tutor Ruggero Maffei con il preziosissimo aiuto del Sig. Adriano Trevisan.

\section{Analisi NginxREScript.ps1}
Nel caso in cui il lancio o rimozione di \textit{sandbox} applicative 
%**************************************************************