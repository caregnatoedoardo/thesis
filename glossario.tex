
%**************************************************************
% Acronimi
%**************************************************************
\renewcommand{\acronymname}{Acronimi e abbreviazioni}

\newacronym[description={\glslink{apig}{Application Program Interface}}]
    {api}{API}{Application Program Interface}

\newacronym[description={\glslink{umlg}{Unified Modeling Language}}]
    {uml}{UML}{Unified Modeling Language}

%**************************************************************
% Glossario
%**************************************************************
%\renewcommand{\glossaryname}{Glossario}

\newglossaryentry{apig}
{
    name=\glslink{api}{API},
    text=Application Program Interface,
    sort=api,
    description={in informatica con il termine \emph{Application Programming Interface API} (ing. interfaccia di programmazione di un'applicazione) si indica ogni insieme di procedure disponibili al programmatore, di solito raggruppate a formare un set di strumenti specifici per l'espletamento di un determinato compito all'interno di un certo programma. La finalità è ottenere un'astrazione, di solito tra l'hardware e il programmatore o tra software a basso e quello ad alto livello semplificando così il lavoro di programmazione}
}

\newglossaryentry{API}
{
    name=\glslink{api}{API},
    text=API,
    sort=api, 
    description={in ingegneria del software \emph{UML, Unified Modeling Language} (ing. linguaggio di modellazione unificato) è un linguaggio di modellazione e specifica basato sul paradigma object-oriented. L'\emph{UML} svolge un'importantissima funzione di ``lingua franca'' nella comunità della progettazione e programmazione a oggetti. Gran parte della letteratura di settore usa tale linguaggio per descrivere soluzioni analitiche e progettuali in modo sintetico e comprensibile a un vasto pubblico}
}

\newglossaryentry{Container}
{
    name=\glslink{container}{CONTAINER},
    text=Container,
    sort=container, 
    description={unità atomica contenente un applicativo con i relativi files atti al corretto funzionamento dello stesso}
}

\newglossaryentry{hardware}
{
    name=\glslink{hardware}{HARDWARE},
    text=hardware,
    sort=hardware, 
    description={insieme delle componenti fisiche di un elaboratore}
}

\newglossaryentry{software}
{
    name=\glslink{software}{SOFTWARE},
    text=software,
    sort=software, 
    description={insieme delle componenti logiche (virtuali) di un elaboratore}
}

\newglossaryentry{Virtual Machine}
{
    name=\glslink{virtual machine}{VIRTUAL MACHINE},
    text=Virtual Machine,
    sort=virtual machine, 
    description={software che, attraverso un ambiente virtuale, emula il comportamento di una macchina fisica con il relativo sistema operativo}
}

\newglossaryentry{Docker}
{
    name=\glslink{docker}{DOCKER},
    text=Docker,
    sort=docker, 
    description={piattaforma software atta alla creazione di applicativi containerizzati}
}

\newglossaryentry{Docker Engine}
{
    name=\glslink{docker engine}{DOCKER ENGINE},
    text=Docker Engine,
    sort=docker engine, 
    description={motore del programma Docker atto alla creazione, esecuzione e gestione di applicativi containerizzati}
}

\newglossaryentry{rollback}
{
    name=\glslink{rollback}{ROLLBACK},
    text=rollback,
    sort=rollback, 
    description={azione di ripristino di una configurazione precedente}
}

\newglossaryentry{open-source}
{
    name=\glslink{open-source}{OPEN-SOURCE},
    text=open-source,
    sort=open-source, 
    description={software non protetto da copyright e liberamente modificabile dagli utenti}
}

\newglossaryentry{NGINX}
{
    name=\glslink{nginx}{NGINX},
    text=NGINX,
    sort=nginx, 
    description={applicativo web-server e load-balancer leggero ad alte prestazioni}
}

\newglossaryentry{Docker API}
{
    name=\glslink{docker API}{DOCKER API},
    text=Docker API,
    sort=docker api, 
    description={set di definizioni e protocolli resi disponibili all'utente da parte del Docker Engine}
}

\newglossaryentry{HDA}
{
    name=\glslink{hda}{HDA},
    text=HDA,
    sort=hda, 
    description={help desk software per la governance automatizzata dei processi del service desk}
}

\newglossaryentry{Docker Compose}
{
    name=\glslink{docker compose}{DOCKER COMPOSE},
    text=Docker Compose,
    sort=docker compose, 
    description={strumento per la definizione e condivisione agevole di applicativi multi-container}
}

\newglossaryentry{CMD}
{
    name=\glslink{cmd}{CMD},
    text=CMD,
    sort=cmd, 
    description={noto anche come Prompt dei comandi, è la shell a riga di comando principale dei sistemi operativo Windows}
}

\newglossaryentry{Powershell}
{
    name=\glslink{Powershell}{POWERSHELL},
    text=Powershell,
    sort=Powershell, 
    description={shell a riga di comando (CLI) sviluppata da Microsoft utilizzabile tramite linguaggio di scripting}
}

\newglossaryentry{CX Studio}
{
    name=\glslink{cx studio}{CX STUDIO},
    text=CX studio,
    sort=cx studio, 
    description={framework multicanale di dialogo proattivo, pensato per le aziende che vogliono attivare e migliorare la propria strategia digitale}
}

\newglossaryentry{client}
{
    name=\glslink{client}{CLIENT},
    text=client,
    sort=client, 
    description={componente che accede ai servizi o alle risorse di un'altra componente, detta server}
}

\newglossaryentry{load-balancing}
{
    name=\glslink{load-balancing}{LOAD-BALANCING},
    text=load-balancing,
    sort=load-balancing, 
    description={capacità di bilanciare e distribuire il carico di lavoro tra diversi server o applicazioni}
}

\newglossaryentry{sistemi operativi}
{
    name=\glslink{sistemi operativi}{SISTEMI OPERATIVI},
    text=sistemi operativi,
    sort=sistemi operativi, 
    description={insieme delle componenti software che garantisce l'operatività e la gestione delle componenti hardware, periferiche o risorse software connesse}
}

\newglossaryentry{Hyper-V}
{
    name=\glslink{hyper-v}{HYPER-V},
    text=Hyper-V,
    sort=hyper-v, 
    description={tecnologia di virtualizzazione sviluppata da microsoft basata su hypervisor}
}

\newglossaryentry{Xen}
{
    name=\glslink{xen}{XEN},
    text=Xen,
    sort=xen, 
    description={ipervisore di macchine virtuali open source}
}

\newglossaryentry{host}
{
    name=\glslink{host}{HOST},
    text=host,
    sort=host, 
    description={dispositivo, solitamente in una rete informatica, nella quale sono memorizzate informazioni}
}

\newglossaryentry{guest}
{
    name=\glslink{guest}{GUEST},
    text=guest,
    sort=guest, 
    description={in informatica, indica il sistema operativo di base}
}

\newglossaryentry{disaster recovery}
{
    name=\glslink{disaster recovery}{DISASTER RECOVERY},
    text=disaster recovery,
    sort=disaster recovery, 
    description={approccio adottato da un'organizzazione per ripristinare la propria infrastruttura informatica a seguito di eventi disastrosi}
}

\newglossaryentry{recovery}
{
    name=\glslink{recovery}{RECOVERY},
    text=recovery,
    sort=recovery, 
    description={procedura di ripristino dei dati}
}

\newglossaryentry{datacenter}
{
    name=\glslink{datacenter}{DATACENTER},
    text=datacenter,
    sort=datacenter, 
    description={cuore tecnologico di ogni azienda informatica}
}

\newglossaryentry{hypervisor}
{
    name=\glslink{hypervisor}{HYPERVISOR},
    text=hypervisor,
    sort=hypervisor, 
    description={processo che crea o gestisce istanze di macchine virtuali}
}

\newglossaryentry{RAM}
{
    name=\glslink{ram}{RAM},
    text=RAM,
    sort=ram, 
    description={acronimo di Random Access Memory, memoria volatile nella quale vengono memorizzate informazioni relative alla sessione di lavoro corrente}
}

\newglossaryentry{storage}
{
    name=\glslink{storage}{STORAGE},
    text=storage,
    sort=storage, 
    description={insieme di supporti di memorizzazione dei dati persistenti}
}

\newglossaryentry{CPU}
{
    name=\glslink{cpu}{CPU},
    text=CPU,
    sort=cpu, 
    description={acronimo di Central Processing Unit, è l'unità di elaborazione centrale di un calcolatore}
}

\newglossaryentry{DMA}
{
    name=\glslink{dma}{DMA},
    text=DMA,
    sort=dma, 
    description={acronimo di Direct Memory Access, dispositivo che permette di accedere direttamente alla memoria RAM da parte di periferiche}
}

\newglossaryentry{NAT}
{
    name=\glslink{nat}{NAT},
    text=NAT,
    sort=nat, 
    description={acronimo di Network Address Translation, è la capacità di un router di tradurre un indirizzo IP pubblico in un indirizzo IP privato e viceversa}
}

\newglossaryentry{VLAN}
{
    name=\glslink{vlan}{VLAN},
    text=VLAN,
    sort=vlan, 
    description={isegmentazione di una rete in sottoreti, gestite tramite apparati di rete quali switch}
}

\newglossaryentry{sandbox}
{
    name=\glslink{sandbox}{SANDBOX},
    text=sandbox,
    sort=sandbox, 
    description={è un ambiente di prova controllato ed isolato dal sistema operativo per testare degli applicativi}
}

\newglossaryentry{filesystem}
{
    name=\glslink{filesystem}{FILESYSTEM},
    text=filesystem,
    sort=filesystem, 
    description={organizzazione logica dei dati su un supporto di memorizzazione}
}

\newglossaryentry{boot}
{
    name=\glslink{boot}{BOOT},
    text=boot,
    sort=boot, 
    description={procedura di accensione di un sistema operativo}
}

\newglossaryentry{ISO}
{
    name=\glslink{iso}{ISO},
    text=ISO,
    sort=iso, 
    description={file che contiene un'immagine avviabile di un sistema operativo}
}

\newglossaryentry{driver}
{
    name=\glslink{driver}{DRIVER},
    text=driver,
    sort=driver, 
    description={insieme delle procedure software che permettono al sistema operativo di pilotare e controllare un dispositivo hardware}
}

\newglossaryentry{guest additions}
{
    name=\glslink{guest additions}{GUEST ADDITIONS},
    text=guest additions,
    sort=guest additions, 
    description={estensione di funzionalità e driver applicativi legati al sistema di virtualizzazione open-source VirtualBox}
}

\newglossaryentry{dockerfile}
{
    name=\glslink{dockerfile}{DOCKERFILE},
    text=dockerfile,
    sort=dockerfile, 
    description={file di testo che permette la costruzione di un'immagine applicativa e del relativo container}
}

\newglossaryentry{caching}
{
    name=\glslink{caching}{CACHING},
    text=caching,
    sort=caching, 
    description={livello di storage temporaneo di dati ad alta velocità e non persistente}
}

\newglossaryentry{layer}
{
    name=\glslink{layer}{LAYER},
    text=layer,
    sort=layer, 
    description={in relazione a Docker, strato di codice di un'immagine applicativa}
}

\newglossaryentry{installer}
{
    name=\glslink{installer}{INSTALLER},
    text=installer,
    sort=installer, 
    description={processo automatizzato volto all'installazione di un applicativo su un computer}
}

\newglossaryentry{BIOS}
{
    name=\glslink{bios}{BIOS},
    text=BIOS,
    sort=bios, 
    description={acronimo di Basic Input Output System, è il primo programma che viene eseguito nella fase di accensione del computer}
}

\newglossaryentry{CNAME}
{
    name=\glslink{cname}{CNAME},
    text=CNAME,
    sort=cname, 
    description={voce all'interno di un albero DNS nella quale viene identificata una risorsa}
}

\newglossaryentry{flow-chart}
{
    name=\glslink{low-chart}{FLOW-CHART},
    text=flow-chart,
    sort=flow-chart, 
    description={anche detto "diagramma di flusso", è una rappresentazione grafica dei passi ed istruzioni di un programma}
}

\newglossaryentry{thread}
{
    name=\glslink{thread}{THREAD},
    text=thread,
    sort=thread, 
    description={suddivisione di un processo in sotto-processi che vengono eseguiti concorrentemente da un processore}
}

